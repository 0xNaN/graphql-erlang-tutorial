\documentclass[lualatex]{beamer}
\usepackage{textcomp}
\usepackage{fontspec}
\usepackage{fancyvrb}
\fvset{fontsize=\scriptsize}
\RecustomVerbatimEnvironment{verbatim}{Verbatim}{}
\usefonttheme{professionalfonts}
\setmainfont{Lucida Grande}
\setsansfont{Lucida Grande}
\usepackage{unicode-math}
\setmathfont{Asana Math}

\author{Jesper Louis
  Andersen\\jesper.louis.andersen@gmail.com\\ShopGun}
\date{\today{}}
\title{GraphQL Erlang}
\begin{document}

\maketitle

\begin{frame}
  Overview:
  \begin{itemize}
  \item Background
  \item GraphQL Itself
  \item The implementation
  \end{itemize}
\end{frame}

\begin{frame}
  \frametitle{Not covered}
  All of these has a story within GraphQL, but I have no time to cover
  them. GraphQL is fairly large:
  \begin{itemize}
  \item Subscriptions
  \item Abstract types: Interfaces, Unions
  \item Authentication/Authorization
  \item Error handling
  \item Schema Loading and Validation
  \item Directives
  \item Aliasing of field names
  \end{itemize}
\end{frame}

\begin{frame}
  \frametitle{Once upon a time\ldots{}}
  \begin{itemize}
  \item ShopGun's mission: index the worlds shopping data
  \item Shopping data: Semi-structured data set
  \item Think Time Zones and Calendars
  \item Densely populated dataset, many links
  \end{itemize}

  How do we create an API for such a data model?
\end{frame}

\begin{frame}
  \frametitle{State}
  We started with some analysis:
  \begin{itemize}
  \item Have existing HTTP/1.1 API
  \item HTTP/1.1 or HTTP/2 ng?
  \item Falcor?
  \item GraphQL?
  \end{itemize}

  Ended with GraphQL: heaviest solution but also solves our problems.
\end{frame}

\begin{frame}
  What are our major problems in the current API?
  \begin{itemize}
  \item Multiple clients: Each client needs different data
  \item Some clients use typed languages, some use untyped languages
  \item Many obvious type errors occur and slows development
  \item The data evolves over time, and requires lots of server-side
    tuning
  \item Documentation is added ad-hoc to the API
  \item Request/Response structure is unclear and client developers
    spend time adapting
  \end{itemize}
\end{frame}

\begin{frame}
  \frametitle{GraphQL: Initial Commit}
  \begin{itemize}
  \item Created by Facebook in 2012, public (draft) spec in 2015
  \item Used on Android (Java), iOS (Obj-C), Web (Javascript)
  \item Can be used to replace (RESTful) web services
  \item Client/Server \emph{Query Language}
  \item Some ideas from Armstrong's UBF are in there
  \item Often JSON output, but isn't bound to JSON
  \end{itemize}
\end{frame}

\begin{frame}
  \frametitle{GraphQL Major features we like}
  \begin{itemize}
  \item There is a schema-definition of data (contract)
  \item The schema is checked for internal consistency (contract checking)
  \item Client \emph{declares} what it wants through \emph{query}
    (declarative)
  \item Client declarations must explicitly mention the data wanted in
    the request
  \item The server handles and processes the queries (query execution)
  \end{itemize}
  \rule{\textwidth}{1pt}
\end{frame}

\begin{frame}
  \begin{itemize}
  \item The schema is fully typed
  \item An request with a (non-coercible) type error is rejected
  \item A response with a type error is \emph{coerced} into a valid
    response
  \item The server allows introspection queries on the meta-structure
    of the contract (automatic discovery)
  \end{itemize}
  \rule{\textwidth}{1pt}
  Note: These things solves our current major problems.
\end{frame}

\begin{frame}[fragile]
  \frametitle{Example}
\begin{columns}[T] % align columns
\begin{column}{.5\textwidth}
\color{black}\rule{\linewidth}{1pt}
Input (GraphQL):
\begin{verbatim}
query Planet {
  node(id: "UGxhbmV0OjE=") {
    ... on Planet {
      id
      name
      orbitalPeriod
    }
  }
}
\end{verbatim}
\end{column}
\hfill
\begin{column}{.5\textwidth}
\color{gray}\rule{\linewidth}{1pt}
Output (JSON):
\begin{verbatim}
{
  "data": {
    "node": {
      "id": "UGxhbmV0OjE=",
      "name": "Tatooine",
      "orbitalPeriod": 304
    }
  }
}
\end{verbatim}
\end{column}%
\end{columns}
\begin{itemize}
\item Only requested fields are returned
\item Must request all fields
\item Output structure reflects input structure
\end{itemize}
\end{frame}

\begin{frame}[fragile]
  \frametitle{Example}
\begin{columns}[T] % align columns
\begin{column}{.5\textwidth}
\color{black}\rule{\linewidth}{1pt}
\begin{verbatim}
query Q {
  room(id: "cm9vbToz") {
    description
    exits {
      direction
      room {
        id
        description
      }
    }
  }
}
\end{verbatim}
\end{column}
\hfill
\begin{column}{.5\textwidth}
\color{gray}\rule{\linewidth}{1pt}
\begin{verbatim}
"room": {
  "description": "Dungeon Entrance",
  "exits": [
   {
    "direction": "north",
    "room": {
     "description": "A dark tunnel",
     "id": "cm9vbTox" } },
   {
    "direction": "secret_passage",
    "room": {
     "description":
       "In a secret passage",
     "id": "cm9vbToy" } }]
}
\end{verbatim}
\end{column}
\end{columns}
\begin{itemize}
\item Schema defines if a field is a scalar or object
\item Schema defines if a field is composite: (array, non-null)
\end{itemize}
\end{frame}

\begin{frame}
  Our current API responds slowly at times, due to the round-trip
  time between the client and the server. Especially on mobile phones
  with bad connectivity.
  
  \begin{itemize}
  \item How do we solve this?
  \end{itemize}
\end{frame}

\begin{frame}
  \begin{itemize}
  \item One query, all operations happen on the server side
  \item Round trip time is between servers, often a few milliseconds
    at most
  \item Lower latency achieved as a result
  \item Can avoid lots of ``boiler plate'' endpoints
  \item Move most ``looping'' in RESTful services to the GraphQL
    execution engine
  \end{itemize}
\end{frame}

\begin{frame}[fragile]
  \frametitle{Fragments}
\begin{verbatim}
query Q {
  monster(id:"...") {
    ...MonsterFrag
  }
  room(id:"...") {
    contents {
      ...MonsterFrag
    }
  }
}

fragment MonsterFrag on Monster {
  id
  name
  hitpoints
}
\end{verbatim}
\begin{itemize}
\item Fragments allow concise reference to fields
\item Fragments also provide ``downcasting'' (contents ``can'' be a
  monster)
\end{itemize}
\end{frame}

\begin{frame}
  \frametitle{Fragments (2)}
  \begin{itemize}
  \item Clients build a fragment for each of their UI elements
  \item Throws every fragment they got at the server
  \item Server performs ``field collection'' to merge the fragments
    into one query
  \item Clients are free to use these features or not
  \item Clients can evolve at different pace
  \end{itemize}
\end{frame}
\begin{frame}[fragile]
  \frametitle{Parameterized queries}
\begin{verbatim}
query Q($monsterId: Id!) {
  monster(id: $monsterId) {
    ...MonsterFrag
  }
}
\end{verbatim}
\begin{itemize}
\item Parameterize Q so it can be reused again and again
\item Query document contains 50-60 queries. You select one query by
  its name and provide its parameters
\item Arguably safer once you lock down the query document in
  production
\item Maximally flexible in development, execution of ``stored
  procedures'' in production
\end{itemize}
\end{frame}

\begin{frame}[fragile]
  \frametitle{Mutations}
\begin{columns}[T] % align columns
\begin{column}{.5\textwidth}
\color{black}\rule{\linewidth}{1pt}
\begin{verbatim}
mutation NewMonster {
  introduceMonster(input:
    {clientMutationId: "123",
     name: "Succubus",
     hitpoints: 24,
     color: "#bbbb00"}) {
    clientMutationId
    monster {
      id
      name
    }
  }
}
\end{verbatim}
\end{column}
\hfill
\begin{column}{.5\textwidth}
\color{gray}\rule{\linewidth}{1pt}
\begin{verbatim}
"introduceMonster": {
  "clientMutationId": "123",
  "monster": {
    "id": "bW9uc3Rlcjoz",
    "name": "Succubus"
  }
}
\end{verbatim}
\end{column}
\end{columns}
\begin{itemize}
\item Changes are through mutations
\item A mutation is \emph{like} a query (but the server handles it differently)
\item Note input objects!
\end{itemize}
\end{frame}

\begin{frame}
  \frametitle{What you have seen until now}
  \begin{itemize}
  \item The queries are from GraphQL test cases
  \item There is a GraphQL server written in Erlang
  \item There is a complete tutorial implementing a database for Star
    Wars\texttrademark{} entities.
  \item The tutorial is backed by an in-memory, disk-backed persistent
    mnesia instance
  \end{itemize}
\end{frame}

\begin{frame}
  DEMO(!!)
\end{frame}

\begin{frame}[fragile]
  \frametitle{Server-side}
  \begin{itemize}
  \item We have a parser for typical GraphQL specifications
  \item You then \emph{map} Erlang modules to schema types
  \item Creates relationship between type and code
  \end{itemize}
\begin{verbatim}
%% In Schema spec:
type Planet implements Node {
  id : ID!
  name : String
  diameter : Int
  rotationPeriod : Int
  orbitalPeriod : Int
  ...
}

%% In Erlang code:
#{ 'Planet' => sw_core_planet, ... }
\end{verbatim}
\end{frame}

\begin{frame}
  \frametitle{Erlang Implementation}
  Insight: The GraphQL system is a programming language
  \begin{itemize}
  \item Turn GQL query documents into (optimized) query plans
  \item Currently about 1/3 of the official de-facto Node.js
    implementation
  \item Almost feature complete
  \item Many other engines use an Object-Oriented visitor pattern
    scheme. We thought we could use a functional approach
  \end{itemize}
  \includegraphics[scale=0.5]{compiler.pdf}
\end{frame}

\begin{frame}
  \frametitle{Lexing and Parsing}
  \begin{itemize}
  \item Standard Erlang lexer generator leex
  \item Could be hand rolled
  \item Not on the critical path
  \end{itemize}
\end{frame}

\begin{frame}
  \frametitle{Elaboration}
  \begin{itemize}
  \item Trick from Standard ML compilers (type inference,
    defunctorization, phase splitting etc)
  \item Elaborate the query by annotating schema types
  \item Makes the later stages far easier to write
  \item Not on the critical path
  \end{itemize}
\end{frame}

\begin{frame}
  \frametitle{Type Check and Validation}
  \begin{itemize}
  \item Fairly standard type checker
  \item Validator steps further verifies query document correctness
    for common mistakes.
  \item Not on the critical path
  \item Note: digression from the spec---Push more things to the type
    checker where it belongs. Push more to the elaborator where it belongs.
  \end{itemize}
\end{frame}

\begin{frame}
  \frametitle{Execution}
  \begin{itemize}
  \item Runs the query
  \item On the critical path!
  \item Uses user-supplied ``resolver'' modules to resolve the
    actual data query.
  \item Resolvers can be backed any code you want
  \item Note: we resolve by modules whereas everyone else resolves by
    functions. (Pattern matching FTW!)
  \end{itemize}
\end{frame}

\begin{frame}[fragile,fragile,fragile]
  \frametitle{Resolver example: Planets}
\begin{verbatim}
execute(_Ctx, #planet { id = PlanetId } = Planet, Field, Args) ->
    case Field of
        <<"id">> -> {ok, sw_core_id:encode({'Planet', Planet#planet.id})};
        <<"edited">> -> {ok, Planet#planet.edited};
        <<"climate">> -> {ok, Planet#planet.climate};
        <<"surfaceWater">> -> {ok, Planet#planet.surface_water};
        <<"name">> -> {ok, Planet#planet.name};
        <<"diameter">> -> {ok, integer(Planet#planet.diameter)};
        <<"rotationPeriod">> -> {ok, integer(Planet#planet.rotation_period)};
        ...;
    end
\end{verbatim}
  Generic resolver:
\begin{verbatim}
execute(_Ctx, Obj, Field, _Args) ->
    {ok, maps:get(Field, Obj, null)}.
\end{verbatim}
\end{frame}

\begin{frame}
  \frametitle{Performance}
  \begin{itemize}
  \item Only parameter checking and execution is time critical
  \item execution, even for large queries are measured in $\mu{}$s,
    usually in the 5-10 range
  \item Fetching data is measured in ms and some times much higher
  \item Your efficiency kernel is likely to be in data fetching
  \item Allows our code to be cleaner as efficiency isn't that
    important
  \item Can really play Erlang's concurrency strength here
  \end{itemize}
\end{frame}

\begin{frame}
  \frametitle{Further Work}
  \begin{itemize}
  \item Introduce a small functional language as the IR
  \item Translate GraphQL to IR, type check IR
  \item Hunch: This is way easier
  \item Type system obviously has modes/polarity in it
  \item Want to formalize type system in Coq/Agda/Twelf at some point
    (Twelf is alluring if we manage to build a $\lambda{}$-calc based IR)
  \item QuickCheck approaches are obvious for testing as well
  \end{itemize}
\end{frame}

\begin{frame}
  \frametitle{Further Work (2)}
  \begin{itemize}
  \item Some validations are still missing
  \item The code is somewhat mature, but has failing corner cases
  \item Concurrent/Parallel query execution is not yet in. Foci:
    correctness first
  \item Some older ideas in the system can be cut out
  \item Build a dedicated handler for Cowboy (awaits Cowboy 2.0)
  \end{itemize}
\end{frame}

\begin{frame}
  \frametitle{Wanna try it?}
  \begin{itemize}
  \item Code is at \url{https://github.com/shopgun/graphql-erlang}
  \item Tutorial is at
    \url{https://github.com/shopgun/graphql-erlang-tutorial}
  \item Tutorial can be viewed at
    \url{https://shopgun.github.io/graphql-erlang-tutorial/}
  \end{itemize}
\end{frame}

\begin{frame}
  QUESTIONS?
\end{frame}

\begin{frame}
  \frametitle{Subscriptions}
  \begin{itemize}
  \item Method to subscribe to updates on an object
  \item Rather new functionality, not yet part of the draft spec
  \item Works just like a mutation, however, trivially implemented
  \end{itemize}
\end{frame}

\begin{frame}
  \frametitle{Authentication}
  \begin{itemize}
  \item Pass around a context to each resolver.
  \item Store Authentication/Authorization info in the context.
  \item Write the resolver such that it inspects the context for auth
    information.
  \item special objects: me, viewer, \ldots.
  \end{itemize}
\end{frame}

\end{document}
% Local Variables:
% TeX-engine: luatex
% TeX-master: t
% End:
